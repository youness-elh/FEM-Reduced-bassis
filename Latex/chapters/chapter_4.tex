\chapter[BONUS and Conclusion]{\uline{BONUS and Conclusion}}

\textbf{Summary} :
In the first part of the problem we had proven that the fin problem has a solution garanted by lax-Milgram. The solution can be deternined by the finite element method but it's time comsuming especially because it is a design problem. So in the linear compliant case for instance, one has to change the parameters set $\mu \in D$ for each iteration at cost $O(\cal N)$. Then, online forming and solving $A_N(\mu)$ at cost $O(Q \cal {N}^2) + O(\cal {N}^3)$. This is why the reduced basis was introduced in part 2 in order to reduce the online computing independent on $\cal N >> N $.

The condition number of a function measures how much the output value of the function can change for a small change in the input argument. This is used to measure how sensitive a function is to changes or errors in the input, and how much error in the output results from an error in the input. Thus the importance of the condition number.

\textbf{Bonus} :

I think the idea is to write $\Omega_i \, i=1,..,4$ as dependant to t and L. In that case, $\Omega_i = [0,t]\times [0,L] \, i=1,..,4$ and $\Omega_0 = [0,t_0]\times [0,L_0]$ and $\Gamma/\Gamma_{root} = [0,C_0]$ all devided by the circonference of the root. We proceed to vraiable change and we obtain an additional multiplication of $t \times L$ included in the affine decomposition equation as follows:
\equaframe{ A_{N}(\mu) =   t_0 \times L_0 \Theta^0(\mu)   A_{N}^0+ \sum_{q=1}^4  t \times L \Theta^q(\mu)   A_{N}^q + circonference_{\Gamma/\Gamma_{root}}\Theta^5(\mu)   A_{N}^5}{}
all devided by the circonference of the root.
